\documentclass[12pt]{beamer}
\usepackage{beamerthemeHannover, graphicx, amsmath, amssymb, multicol}
\usepackage{textcomp}
\usepackage{verbatim}
\usepackage{listings}
\setbeamercolor{sidebar}{use=structure,bg=gray!60!green}
\lstset{language=SQL}

\title{PL/Perl6 \\ \small {Butterflies in your PostgreSQL database} }
\author[@dukeleto]{Jonathan "Duke" Leto}
\date{}

\begin{document}

\frame{
    \titlepage
    \begin{columns}[tc]
            \includegraphics[width=121px, height=112px]{camelia}
            \includegraphics[width=101px, height=135px]{parrot}
    \end{columns}
}

\frame{
    \frametitle{Parrot Virtual Machine}
    \begin{center}
        \begin{itemize}
            \item Process (Application) Virtual Machine
            \item Register-based
            \item Continuation Passing Style
            \item Design Goals
            \begin{itemize}
                \item Pluggable
                \item Interoperable
                \item Dynamic
            \end{itemize}
            \item 3.6.0 just released
        \end{itemize}
    \end{center}
}
\frame{
    \frametitle{Rakudo Perl 6}
    \begin{center}
        \begin{itemize}
            \item Most active implementation of Perl 6
            \item Currently uses Parrot as a backend
            \item Stable release series is Rakudo Star
            \item New Rakudo Star expected this summer
        \end{itemize}
    \end{center}
}

\frame{
    \frametitle{Why Embed Parrot VM in PostgreSQL?}
    \begin{center}
        \begin{itemize}
            \item Procedural/PostgreSQL Languages (PL's) are hard to write and maintain
            \item PL/Parrot does the hard work, HLLs benefit with much less effort
            \item Platform independent, fast$^\dagger$, stored procedures
        \end{itemize}
    \end{center}
}

\frame{
    \frametitle{PL/Parrot Timeline}
    \begin{center}
        \begin{itemize}
            \item PostgreSQL came from Ingres, started in the mid 1970's
            \item Parrot started around 2001
            \item Empty pgFoundry repo created 2006 by David Fetter/Joshua Tolley
            \item GitHub repo created Oct 2009 by Duke Leto
            \item PL/Perl6 started to work June 2010
            \item New Embed API in Parrot implemented Late 2010
            \item Currently translating to the new API
        \end{itemize}
    \end{center}
}

\frame{
    \frametitle{Current Features}
    \begin{itemize}
        \item PL/PIR and PL/Perl6
        \item Pass and return basic datatypes
        \item Basic security model (Don't do that)
        \item Growing Test Suite
        \item Enthusiastic and friendly community
        \item Coolest Feature: Use Perl 6 grammars in PostgreSQL!
    \end{itemize}
}

\frame{
    \frametitle{Things that creak}
    \begin{itemize}
        \item Documentation - http://pl.parrot.org
        \item Doesn't compile on latest Parrot due to bitrot
        \item SPI - branch being worked by cxreg++, elog works!
        \item Triggers - we need help!
        \item SETOF - branch with some tests
        \item Row handling
        \item Parrot Bugs
        \begin{itemize}
            \item IMCC Syntax Errors
            \item Security API
            \item Extend/Embed API
        \end{itemize}
    \end{itemize}
}

\frame{
    \frametitle{Installing/Testing PL/Parrot}
    \begin{itemize}
        \item Install Rakudo or Parrot
        \item \# parrot\_config needs to be in your \textdollaroldstyle PATH
        \item git clone git://github.com/leto/plparrot.git
        \item cd plparrot
        \item export PGPORT=5555 \# if necessary
        \item make install installcheck \# might need sudo
        \item make test test\_plperl6 \# PL/PIR + PL/Perl6 tests
    \end{itemize}
    NEW: Get it from PGXN.org! http://pgxn.org/dist/plparrot/
}

\frame{
    \frametitle{PL/PIR Example 1}
    \small{
        \lstinputlisting{pir1.sql}
    }
}

\frame{
    \frametitle{PL/Perl6 Example 1}
    \small{
        \lstinputlisting{p6fib.sql}
    }
}

\frame{
    \frametitle{PL/Perl6 Example 2}
    \tiny{
        \lstinputlisting{p6grammar.sql}
    }
}

\frame{
    \frametitle{How is PL/PIR Sausage Made?}
    \begin{itemize}
        \item A Parrot interpreter is created for each language
        \item PG datatypes are converted to Parrot datatypes, in C, with magic
        \item Stored procedure code is wrapped in anonymous sub, compiled and invoked from C
        with Parrot\_ext\_call
        \item Parrot datatypes are converted back to PG datatypes
    \end{itemize}
}

\frame{
    \frametitle{How is PL/Perl6 Sausage Made?}
    \begin{itemize}
        \item perl6.pbc is automagically found and loaded into the interp for PL/Perl6
        \item PG datatypes are converted to PL/Perl6 datatypes in PIR with \&infix\textless,\textgreater
        \item Stored procedure code is wrapped in a closure, compiled and invoked from PIR
        \item PL/Perl6 datatypes are converted back to PG datatypes
    \end{itemize}
}

\frame{
    \frametitle{Current Goals}
    \begin{itemize}
        \item Convert to new Parrot Embed API
        \item Easy onramps to add new languages to PL/Parrot
        \item Framework for DSL's
        \item Allow various PL's to communicate
        \item Freeze/thaw subtransaction-level states
    \end{itemize}
}

\frame{
    \frametitle{Get involved!}
    \begin{itemize}
        \item Try PL/Parrot on your system and submit detailed bug reports
        \item Fork on github and hack on stuff!
        \item Help with GitHub Issues
        \item \#plparrot on freenode
        \item http://pl.parrot.org
        \item http://groups.google.com/group/plparrot
    \end{itemize}
}

\frame{
    \frametitle{ Thanks }
    \begin{itemize}
        \item PL/Parrot team:
            \begin{itemize}
            \item David Fetter, David E. Wheeler, Joshua Tolley, Daniel Arbelo Arrocha, Dave Olszewski
            \item AKA davidfetter++, theory++, eggyknap++, darbelo++, cxreg++
            \end{itemize}
        \item Everyone working on Parrot Virtual Machine, Perl 6 and PostgreSQL
        \item Especially (Moritz Lenz) moritz++, (Peter Lobsinger) plobsing++ and (Stephen O'Rear) sorear++, for great advice and help
    \end{itemize}
}

\frame{
    \frametitle{ Resources }
    \begin{center}
        \begin{itemize}
           \item http://pl.parrot.org
           \item http://github.com/leto/plparrot
           \item http://parrot.org
           \item @parrotvm / !parrot on twitter/identi.ca
           \item @dukeleto / !leto on twitter/identi.ca
           \item Slides available at http://leto.github.com
        \end{itemize}
    \end{center}
}
\end{document}
